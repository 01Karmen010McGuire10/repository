% 
% Sample tex source file
% 

% Use following settings for LaTeX2e.
%\documentclass[times, 10pt,twocolumn]{article} 
%\usepackage{latex8}
%\usepackage{times}

% Use following settings for LaTeX 2.09.
\documentstyle[times,art10,twocolumn,latex8]{article}

% Next nerenvironment margin of the text.
\setlength{\oddsidemargin}{-1in}
\addtolength{\oddsidemargin}{2.5cm}
\setlength{\evensidemargin}{-1in}
\addtolength{\evensidemargin}{2.5cm}
\setlength{\topmargin}{-1in}
% \addtolength{\topmargin}{2.5cm}
\addtolength{\topmargin}{2.2cm}
\setlength{\headheight}{0cm}
\setlength{\headsep}{0cm}
\setlength{\textwidth}{16.0cm}
\setlength{\textheight}{24.2cm}
\setlength{\parindent}{0.4cm}
\setlength{\columnsep}{0.8cm}

%------------------------------------------------------------------------- 
% take the % away on next line to produce the final camera-ready version 
\pagestyle{empty}

%------------------------------------------------------------------------- 
\begin{document}

\title{Author Guidelines for Formal Proceedings of DC-2001 Conference}

\author{DC-2001 Local Arrangement Comittee\\
National Institute of Informatics\\ 
2-1-2 Hitotsubashi, Chiyoda-ku, Tokyo 101-8430, Japan\\
dc2001-local@nii.ac.jp
%%%Please use followings to add another author from different institute.
%%%\and
%%%Second Author\\
%%%Institution2\\
%%%First line of institution2 address\\ Second line of institution2 address\\ 
%%%SecondAuthor@institution2.com\\
}

\maketitle
\thispagestyle{empty}

\begin{abstract}
  The ABSTRACT should be written in fully-justified text, at the top of
  the left-hand column, below the author and affiliation
  information. Use the word ``Abstract'' as the title, in 12-point
  Times, boldface type, centered relative to the column, initially
  capitalized. The abstract is to be in 10-point, single-spaced
  type. The abstract may be up to 7.5 cm (approximately 3 inches) long. 
  KEYWORDS are to be followed by no blank line after the abstract,
  accompanied with the hedding ``Keywords:'' in Times 12-point
  boldface, initially capitalized.
  Leave two blank lines after the Keywords, then begin the main text.\\
{\bf Keywords:} DC-2001, Author Guidelines. 
\end{abstract}



%------------------------------------------------------------------------- 
\section{Introduction}

Please follow the steps outlined below when submitting your manuscript
to the Proceedings of DC-2001 Conference. This guideline and
manuscript are a modified version of the IEEE proceedings
ones\cite{latex8}.

%------------------------------------------------------------------------- 
\section{Instructions}

Please read the following carefully.

%------------------------------------------------------------------------- 
\subsection{Language}

All manuscripts must be in English.

%------------------------------------------------------------------------- 
\subsection{Printing your paper}

Print your properly formatted text on high-quality A4 white printer
paper. 
Those of you who usually use US Letter paper, bear with us---this
layout is for A4 paper, hence the longer-than-usual pages. Adjust the
top-margin if necessary to avoid text running off the bottom of the
page.
% Do not exceed 8 pages for one task (or 12 pages for Chinese IR Tasks
% and Japanese \& English IR Tasks together).
Do not exceed 2 to 4 pages for short paper submission or 
8 to 10 pages for regular paper submission.

%------------------------------------------------------------------------- 
\subsection{Margins and page numbering}

All printed material, including text, illustrations, and charts, must
be kept within a print area 16.0 cm (6-3/8 inches) wide by 24.2 cm
(9-1/2 inches) high. Do not write or print anything outside the print
area.  Number your pages lightly, in pencil, on the upper right-hand
corners of the BACKS of the pages (for example, 1/8, 2/8, or 1 of 8, 2
of 8, and so forth). Please do not write on the fronts of the pages,
nor on the lower halves of the backs of the pages.

%------------------------------------------------------------------------ 
\subsection{Formatting your paper}

Body text must be in a two-column format. The total allowable width of
the text area is 16.0 cm (6-5/16 inches) wide by 24.2 cm (9-1/2
inches) high. Columns are to be 7.6 cm (3.0 inches) wide, with a 0.8 cm
(5/16 inch) space between them. The main title (on the first page)
should begin 3.5 cm (1-3/8 inches) from the top edge of the page. The
second and following pages should begin 2.5 cm (1.0 inch) from the top
edge. On all pages, the bottom margin should be 3.0 cm (1-3/16 inch)
from the bottom edge of the page.

%------------------------------------------------------------------------- 
\subsection{Type-style and fonts}

Wherever Times is specified, Times Roman may also be used. If neither is 
available on your word processor, please use the font closest in 
appearance to Times that you have access to.

MAIN TITLE. Center the title 3.5 cm (1-3/8 inches) from the top edge of 
the first page. The title should be in Times 14-point, boldface type. 
Capitalize the first letter of nouns, pronouns, verbs, adjectives, and 
adverbs; do not capitalize articles, coordinate conjunctions, or 
prepositions (unless the title begins with such a word). Leave two blank 
lines after the title.

AUTHOR NAME(s), AFFILIATION(s), E-MAIL ADDRESS(s) are to be centered
beneath the title and printed in Times 12-point, non-boldface
type. This information is to be followed by two blank lines.

The ABSTRACT, KEYWORDS and MAIN TEXT are to be in a two-column format. 

MAIN TEXT. Type main text in 10-point Times, single-spaced. Do NOT use 
double-spacing. All paragraphs should be indented 1 pica (approx. 
0.4 cm or 1/6 inch). Make sure your text is fully justified---that is, 
flush left and flush right. Please do not place any additional blank 
lines between paragraphs. Figure and table captions should be 10-point 
Helvetica boldface type as in
\begin{figure}[h]
   \caption{Example of caption.}
\end{figure}

\noindent Long captions should be set as in 
\begin{figure}[h] 
   \caption{Example of long caption requiring more than one line. It is 
     not typed centered but aligned on both sides and indented with an 
     additional margin on both sides of 1~pica.}
\end{figure}

\noindent Callouts should be 9-point Helvetica, non-boldface type. 
Initially capitalize only the first word of section titles and first-, 
second-, and third-order headings.

FIRST-ORDER HEADINGS. (For example, {\large \bf 1. Introduction}) 
should be Times 12-point boldface, initially capitalized, flush left, 
with one blank line before, and one blank line after.

SECOND-ORDER HEADINGS. (For example, {\elvbf 1.1. Database elements}) 
should be Times 11-point boldface, initially capitalized, flush left, 
with one blank line before, and one after. If you require a third-order 
heading (we discourage it), use 10-point Times, boldface, initially 
capitalized, flush left, preceded by one blank line, followed by a period 
and your text on the same line.

%------------------------------------------------------------------------- 
\subsection{Footnotes}

Please use footnotes sparingly%
\footnote
   {%
     Or, better still, try to avoid footnotes altogether.  To help your 
     readers, avoid using footnotes altogether and include necessary 
     peripheral observations in the text (within parentheses, if you 
     prefer, as in this sentence).
   }
and place them at the bottom of the column on the page on which they are 
referenced. Use Times 8-point type, single-spaced.


%------------------------------------------------------------------------- 
\subsection{References}

List and number all bibliographical references in 9-point Times, 
single-spaced, at the end of your paper. When referenced in the text, 
enclose the citation number in square brackets, for example~\cite{ex1}. 
Where appropriate, include the name(s) of editors of referenced books.

%------------------------------------------------------------------------- 
\subsection{Illustrations, graphs, and photographs}

All graphics should be centered. Your artwork must be in place in the 
article (preferably printed as part of the text rather than pasted up). 
If you are using photographs and are able to have halftones made at a 
print shop, use a 100- or 110-line screen. If you must use plain photos, 
they must be pasted onto your manuscript. Use rubber cement to affix the 
images in place. Black and white, clear, glossy-finish photos are 
preferable to color. Supply the best quality photographs and 
illustrations possible. Penciled lines and very fine lines do not 
reproduce well. Remember, the quality of the book cannot be better than 
the originals provided. Do NOT use tape on your pages!

%------------------------------------------------------------------------- 
\subsection{Color}

The use of color on interior pages (that is, pages other 
than the cover) is prohibitively expensive. We publish interior pages in 
color only when it is specifically requested and budgeted for by the
conference organizers. DO NOT SUBMIT COLOR IMAGES IN YOUR 
PAPERS UNLESS SPECIFICALLY INSTRUCTED TO DO SO.

%------------------------------------------------------------------------- 
\subsection{Symbols}

If your word processor or typewriter cannot produce Greek letters, 
mathematical symbols, or other graphical elements, please use 
pressure-sensitive (self-adhesive) rub-on symbols or letters (available 
in most stationery stores, art stores, or graphics shops).

%------------------------------------------------------------------------- 
\subsection{Conclusions}

Please direct any questions to the production editor in charge of
these proceedings at the DC-2001 conference organization: E-mail,
$\langle$dc2001-local@nii.ac.jp$\rangle$.

When preparing your manuscript, please see the following web site,
$\langle$http://www.nii.ac.jp/dc2001/$\rangle$.  If you use
'latex8.sty', please adjust the margin settings, the same as
this \LaTeX\ source file of author guidelines.

%------------------------------------------------------------------------- 
\nocite{ex1,ex2}
\bibliographystyle{latex8}
\bibliography{dc2001_guideline}

\end{document}
